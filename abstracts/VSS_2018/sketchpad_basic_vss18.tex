\documentclass{article}
% General document formatting
\usepackage[margin=0.7in]{geometry}
\usepackage[parfill]{parskip}
\usepackage[utf8]{inputenc}
\usepackage{authblk}
\usepackage{palatino}
\usepackage[none]{hyphenat}

% Related to math
\usepackage{amsmath,amssymb,amsfonts,amsthm}

\title{Contextual flexibility in visual communication}
\author[a]{Judith E. Fan}
\author[a]{Robert X.D. Hawkins} 
\author[b]{Mike Wu}
\author[a,b]{Noah D. Goodman}

\affil[a]{Department of Psychology, Stanford University}
\affil[b]{Department of Computer Science, Stanford University}

\begin{document}
\maketitle

Visual modes of communication are ubiquitous in modern life --- from maps to data plots to political cartoons. Here we investigate drawing, the most basic form of visual communication. Communicative drawing poses a core challenge for theories of how vision and social cognition interact, requiring a detailed understanding of how sensory information and social context jointly determine what information is relevant to communicate. Participants (N=192) were paired in an online environment to play a drawing-based reference game. On each trial, both participants were shown the same four objects, but in different locations. The \textit{sketcher's} goal was to draw one of these objects --- the target --- so that the \textit{viewer} could pick it out from a set of distractor objects. There were two types of trials: \textit{close}, where objects belonged to the same category, and \textit{far}, where objects belonged to different categories. We found that people exploited information in common ground with their partner to efficiently communicate about the target: on far trials, sketchers achieved 99.7\% recognition accuracy while applying fewer strokes, using less ink, and spending less time (\textit{p}s$<$0.001) on their drawings than on close trials, where accuracy was still high (87.7\%). We hypothesized that humans succeed at this task by recruiting two core competencies: (1) \textbf{visual abstraction}, the capacity to perceive the correspondence between an object and a drawing of it; and (2) \textbf{social reasoning}, the ability to infer what information would help a viewer distinguish the target from distractors. We instantiated these competencies in a sketcher model that combines a multimodal convnet visual encoder with a Bayesian model of recursive social reasoning, and found that it fit the data well and outperformed a baseline model that ignored the context. Together, this work provides the first unified computational theory of how perception and social cognition support contextual flexibility in real-time visual communication.


\vspace{3mm}

\begin{description}  
\item Word Count: $\sim$ 300 of 300
\item Methodology/Approach: Behavior/Psychophysics
\item Primary Topic Descriptor: Perception and action: other
\item Secondary Topic Descriptor: XX
\item Funding sources: XXX
\item Presentation Preference: XXX
\item Suggested Reviewer 1: XXX
\item Suggested Reviewer 2: XXX
\item Suggested Reviewer 3: XXX
\end{description}


\end{document}