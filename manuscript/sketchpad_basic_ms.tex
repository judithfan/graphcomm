\documentclass[9pt,twocolumn,twoside]{pnas-new}
% Use the lineno option to display guide line numbers if required.
% Note that the use of elements such as single-column equations
% may affect the guide line number alignment.

\templatetype{pnasresearcharticle} % Choose template
% {pnasresearcharticle} = Template for a two-column research article
% {pnasmathematics} = Template for a one-column mathematics article
% {pnasinvited} = Template for a PNAS invited submission

\usepackage{pslatex}
\usepackage{amsfonts}
\usepackage{graphicx}
\usepackage{color}
\usepackage{todonotes}
\usepackage{dsfont}
\usepackage{array}
\usepackage{textcomp}
\usepackage{multirow}
\usepackage{subfig}


\title{Contextual flexibility in visual communication}

% Use letters for affiliations, numbers to show equal authorship (if applicable) and to indicate the corresponding author

\author[a,1]{Judith E. Fan}
\author[a]{Robert X.D. Hawkins}
\author[b]{Mike Wu}
\author[a,b]{Noah D. Goodman}

\affil[a]{Department of Psychology, Stanford University}
\affil[b]{Department of Computer Science, Stanford University}

% Please give the surname of the lead author for the running footer
\leadauthor{Fan}

\significancestatement{}

% Please include corresponding author, author contribution and author declaration information
\authorcontributions{Please provide details of author contributions here.}
\authordeclaration{Please declare any conflict of interest here.}
\correspondingauthor{\textsuperscript{1}To whom correspondence should be addressed. E-mail: jefan@stanford\@email.edu}

% Keywords are not mandatory, but authors are strongly encouraged to provide them. If provided, please include two to five keywords, separated by the pipe symbol, e.g:
\keywords{drawing $|$ deep learning $|$ pragmatics $|$ computational modeling $|$ Rational Speech Act framework}

\begin{abstract}

Visual modes of communication are ubiquitous in modern life --- from maps to data plots to political cartoons. Here we investigate drawing, the most basic form of visual communication. Communicative drawing poses a core challenge for theories of how vision and social cognition interact, requiring a detailed understanding of how sensory information and social context jointly determine what information is relevant to communicate. Participants (N=192) were paired in an online environment to play a drawing-based reference game. On each trial, both participants were shown the same four objects, but in different locations. The \textit{sketcher's} goal was to draw one of these objects --- the target --- so that the \textit{viewer} could pick it out from a set of distractor objects. There were two types of trials: \textit{close}, where objects belonged to the same category, and \textit{far}, where objects belonged to different categories. We found that people exploited information in common ground with their partner to efficiently communicate about the target: on far trials, sketchers achieved 99.7\% recognition accuracy while applying fewer strokes, using less ink, and spending less time (\textit{p}s$<$0.001) on their drawings than on close trials. We hypothesized that humans excel at this task by recruiting two core competencies: (1) \textbf{visual abstraction}, the capacity to perceive the correspondence between an object and a drawing of it; and (2) \textbf{social reasoning}, the ability to infer what information would help a viewer distinguish the target from distractors. We instantiated these competencies in a computational model that combines a multimodal convnet visual encoder with a Bayesian model of recursive social reasoning, and found that it fit the data well and outperformed lesioned variants of the model. Together, this work provides the first unified computational theory of how perception and social cognition support contextual flexibility in real-time visual communication.
\end{abstract}

\dates{This manuscript was compiled on \today}
\doi{\url{www.pnas.org/cgi/doi/10.1073/pnas.XXXXXXXXXX}}


\begin{document}

% Optional adjustment to line up main text (after abstract) of first page with line numbers, when using both lineno and twocolumn options.
% You should only change this length when you've finalised the article contents.
\verticaladjustment{-2pt}

\maketitle
\thispagestyle{firststyle}
\ifthenelse{\boolean{shortarticle}}{\ifthenelse{\boolean{singlecolumn}}{\abscontentformatted}{\abscontent}}{}

Communication is central to the success of our species: it allows us to learn from each other, coordinate our actions, and express otherwise hidden thoughts. People communicate in many ways, including spoken language and gesture. Critically, human communication goes beyond vocal production and manual signing: A watershed moment in the history of human communication was the invention of graphical representation, independently in Europe and Asia, about 30-60 thousand years ago \cite{hoffmann2018u,Aubert:2014jy}.

Graphical representation was transformative because it provided a means for people to encode their thoughts in a durable and shareable format. From ancient etchings on cave walls to modern digital displays, using graphical representations to communicate lies at the heart of key human innovations (e.g., mapmaking, data visualization), and forms the foundation for the cultural transmission of knowledge and higher-level reasoning. Drawing is a particularly important case study for understanding human visual communication. Drawn images predate symbolic writing systems \cite{clottes2008cave} are pervasive in many cultures \cite{gombrich1989story}, and are produced prolifically by children from an early age \cite{kellogg1969analyzing}.

%%%%%%%%%%%%%%%%%%%%

Communicative drawing poses a key challenge for understanding how core cognitive systems are coordinated to support complex, natural behaviors.

Successful visual communication by drawing inherently combines the activity of three such core systems: \textit{visual perception} --- the set of computations that transform raw visual inputs into semantically meaningful internal representations; \textit{social reasoning} --- our robust capacity to evaluate other people's beliefs, intentions, and knowledge; and \textit{action planning} --- the ability to compose sequences of actions to efficiently achieve task goals.

Drawing thus provides a natural domain to interrogate how visually-guided actions are constrained by high-level goals, including social communication, and likewise, how communicative behavior is constrained by our perceptual system.

A long-standing debate concerns the relative contribution of these systems for explaining how drawings derive their referential meaning. On the one hand, there is strong evidence that the perceptual system is sufficient to explain how people are able to perceive the correspondence between drawings and real-world objects, suggesting that this ability can arise from the same general-purpose neural architecture evolved to handle natural visual inputs
\cite{fan2015common, yamins2014performance}. On the other hand, several influential theorists have argued that socially-mediated experience with pictorial representations is always necessary to be able to extract semantic content from drawings \cite{goodman1976languages,gombrich1989story}.

How can these two opposing perspectives on the basis for fluency with graphical representations be reconciled? And how does such variation in visual communication behavior, ranging from the depictive to the symbolic, arise from the same cognitive architecture? Here we propose a unified theoretical framework for understanding how such variation naturally arises from variation in context, affecting how perception and social inference are combined to generate communicative actions.


Here we propose an integrated computational model of visual communication that combines high-performing models of sensory representation from deep learning with insights from Bayesian cognitive models of social reasoning in language.

This  into the core computations that connect perception, action, and social cognition to support visual communication.

%%%%%%%%%%%%%%%%%%%%

The goal of this paper is to evaluate the hypothesis that contextual flexibility in human visual communication can be captured by combining constraints from perception and pragmatic reasoning.

Specifically, are \emph{pragmatic visual communicators} who proactively account for their viewer's expectations and take advantage of shared knowledge to be informative
\cite{wilson1986relevance,goodman2016pragmatic}
while also being parsimonious \cite{zipf1936psycho}.




\section*{Results}



\subsection*{Communication task performance}

We hypothesized that sketchers aim to be informative yet parsimonious in communicating about the target object in context.

We evaluated several predictions that this hypothesis makes:

first, that sketchers are generally successful at conveying the identity of the target to viewers in both close and far contexts.

second, that sketchers will modulate the level of detail in their drawings according to the similarity of the alternatives under consideration by the viewer.

If sketchers aim to make their sketches informative without being too costly to produce, they will modulate the level of detail in their drawings according to the set of alternatives under consideration by the viewer.

For instance, in the \textit{close} condition where the target and distractors belong to the same basic-level category, sketchers should make detailed drawings to highlight fine-grained distinctions, even though this requires including additional detail. By contrast, in the \textit{far} condition where the target and distractors belong to different basic-level categories, the sketcher can afford to produce simpler drawings by omitting excess detail.

Consistent with this prediction, we found that sketchers applied fewer strokes (close = 8.03, far = 13.5, \textit{p} $<$ 0.001), used less ink (measured by proportion of the canvas marked: close = 0.054, far = 0.042, \textit{p} $<$ 0.001) and spent less time (close = 30.3s, far = 13.7s, \textit{p} $<$ 0.001) to make their drawings in the far condition than in the close condition.

Despite the relative sparsity of drawings produced on far trials, viewers were close to ceiling on inferring the target (99.7\%). Performance was also high on close trials (87.7\%), even though this required that people discriminate the correspondence between a sketch and its referent in the context of highly similar distractors.

We hypothesized that the context manipulation in the communication experiment systematically affected the high-level perceptual content in participants' sketches, resulting in Far sketches that were highly informative about the target \textit{in context} yet less recognizable (i.e., worse match to the target object) than Close sketches \textit{out of context}.

In other words, we predicted that the decreased effort expended by participants on Far trials reduced the absolute perceptual similarity between sketches and the target. The goal of this experiment was to validate this assumption and to provide independent training data to train a shallow adaptor network to merge photorealistic images and sketches into a common latent space, from which perceptual similarity can be easily read out.

%%% recognition experiments
%% report close vs. far Accuracy

Consistent with our prediction, we found that Close sketches were matched with their corresponding object rendering much more frequently than Far sketches were (Close: 54.2\%; Far: 37.5\%; $Z$=14.1, $p$<0.001), although both were successfully matched at rates far exceeding chance ($p$s < 0.001).

%% report variation in different categories

%% report confusion matrix results: diagonal vs. off-diagonal block vs. rest of the matrix results.

We further hypothesized that the particular way in which Far sketches would differ perceptually from Close sketches was that Far sketches would be more confusable with other objects from the same basic-level category, while still being highly recognizable as a representation of the corresponding basic-level category.

We measured this by comparing the within-category confusion rate to the overall confusion rate for Close and Far sketches, and found that XXX.

\subsection*{Modeling}

%% describe the full sketcher model

%% report RSA modeling results using human perceptual measurements directly to compute informativity

%% 4 versions: prag with cost, prag no cost, no prag cost, no prag no cost

%% report generalization performance of visual adaptor on the appropriate splits

%% report RSA modeling results, where we apply the same 2 pragmatics (prag vs. no prag) x 2 production (cost vs. no cost) x 2 perception (high layer vs. low layer) comparison categories

\subsubsection*{Code availability} The code for the analyses presented in this article is publicly available in a Github repository at: XX.

\subsubsection*{Data availability} The data presented in this article are publicly available at this URL: XX.
ras
\subsection*{Supporting Information (SI)}

% The main text of the paper must stand on its own without the SI. Refer to SI in the manuscript at an appropriate point in the text. Number supporting figures and tables starting with S1, S2, etc. Authors are limited to no more than 10 SI files, not including movie files. Authors who place detailed materials and methods in SI must provide sufficient detail in the main text methods to enable a reader to follow the logic of the procedures and results and also must reference the online methods. If a paper is fundamentally a study of a new method or technique, then the methods must be described completely in the main text. Because PNAS edits SI and composes it into a single PDF, authors must provide the following file formats only.

\matmethods{

\subsection*{Communication experiment: Manipulation of context in sketch-based reference game}

\subsubsection*{Participants}

A total of 192 unique participants, who were recruited via Amazon Mechanical Turk (AMT) and grouped into pairs, completed the experiment. They were provided a base compensation of \$X.XX for participation and earned a \$X.XX bonus for each correct trial. In this and subsequent behavioral experiments, participants provided informed consent in accordance with the Stanford IRB.

\subsubsection*{Stimuli}

Stimuli were 32 3D mesh models of objects belonging to 4 categories (i.e., birds, chairs, cars, dogs), containing eight objects each. 40 color images of each object were produced by rendering it from a 10$^{\circ}$ viewing angle (i.e., slightly above) at a fixed distance on a gray background, each rotated by an additional 9$^{\circ}$ about the vertical axis.

\subsubsection*{Task}

Drawings were collected in the context of an online, sketching-based reference game (``Guess My Sketch!''). The game involved two players: a \textit{sketcher} who aims to help a \textit{viewer} pick out a target object from a set of distractor objects by representing it in a sketch. On each trial, both participants were shown an array of the same four objects; however, the positions of these objects were randomized for each participant. On each trial, one of the four objects was highlighted on the sketcher's screen to designate it as the target.

Sketchers drew using black ink on digital canvas (pen width = 5 pixels; 300 x 300 pixels) embedded in a web browser window using Paper.js (http://paperjs.org/). Participants drew using the mouse cursor, and were not able to delete previous strokes. Each stroke of which was rendered on the viewer's screen immediately upon the completion of each stroke. There were no restrictions on how long participants could take to make their drawings. After clicking a submit button, the viewer guessed the identity of the drawn object by clicking one of the four objects in the array. Otherwise, the viewer had no other means of communicating with the sketcher. Both participants received immediate task-related feedback: the sketcher learned which object the viewer had clicked, and the viewer learned the identity of the target. Both participants earned bonus points for each correct response.

For each pair, objects were grouped into eight quartets: Four of these quartets contained objects from the same category (``close''); the other four of these quartets contained objects from different categories (``far'' condition). Each quartet was presented four times, such that each object in the quartet served as the target exactly once. The assignment of objects to quartet and condition was randomized across pairs.

\subsection*{Recognition experiment: Measuring perceptual similarity between sketches and objects}

\subsubsection*{Participants}

A total of 112 participants were recruited via Amazon Mechanical Turk (AMT). They were provided a base compensation of \$1.00 for their participation, and earned an additional \$0.01 bonus for each correct response.

\subsubsection*{Task}

On each trial, participants were presented with a randomly selected sketch collected in the sketch-based reference game, surrounded by a grid containing the 32 objects from that experiment. Their goal was to select the object in the grid that best matched the sketch. Participants were instructed to prioritize accuracy over speed. A small proportion of outlier trials that were either too fast (response latency <1000ms) or too slow (>20s) were filtered from the dataset; we reasoned that behavior on such trials was likely to be noisier and not comparable to that on the remainder of trials. The removal of these outlier trials did not have a substantial impact on the pattern of recognition behavior. In order to mitigate the possibility that participants could adjust their matching strategy according to any particular sketcher's style, each session was populated with 64 sketches sampled randomly from different reference games. Participants were provided with binary feedback about the correctness of their response on each trial via a bonus counter that incremented by 1 point for each correct identification, but did not change for incorrect trials.

% VERIFY THIS: Participants were permitted to complete multiple sessions of this task, but were prevented from providing identification judgments for the same sketch twice, or for sketches they themselves had produced or viewed (on the rare occasion that this participant had also participated in the reference game experiment).

\subsection*{Computational modeling}


\subsubsection*{Visual encoder module}

Features were extracted using VGG-19 \cite{simonyan2014very}, a high-performing deep convolutional neural network model architecture that had been pretrained to categorize objects on the Imagenet database, which contains millions of photographs from hundreds of object categories \cite{deng2009imagenet}. Despite the fact that drawings are highly abstracted away from natural visual inputs, prior work has shown a striking isomorphism in the similarity structure \cite{kriegeskorte2008matching} of object categories in drawings and photos at higher layers in these models \cite{fan2015common}, without any additional training.

While these models excel at (and were indeed optimized for) categorization of objects, two additional competencies are required to succeed at the visual communication task described above.

First, an observer must be able to represent fine-grained relationships between different images of objects from the same category; second, this observer must also be able to discern image-level correspondences between drawings and photos of the same object.

In order to better approximate the granularity with which human observers can distinguish different images, we augment VGG19 with a smaller \textit{adaptor} network that learns a common feature representation for drawings and photos.

This approach has been successfully used in prior work to better predict human semantic categorization judgments \cite[]{peterson2016adapting}, and to improve sketch-based image retrieval \cite[]{sangkloy2016sketchy}.

%% adaptor architecture
The purpose of the visual encoder is to learn a function from images and sketches to a shared embedding space. The encoder network takes as input mid-level convolutional features from VGG-19. In our experiments, we used the second convolutional layer after the third pooling step. We chose a mid-convolutional layer to capitalize on successive layers of abstraction from early convolutional layers and to preserve two-dimensional structure. The encoder produces a shared embedding of size 1000.

The architecture of the encoder is as follows: a single convolutional layer (conv1) with 64 filters, a kernel size of 3, a padding of 1, and a stride of 1; followed by a max pool layer with a stride of 2 and a dilation of 1; followed by two fully connected layers (fc1, fc2) to project to 4096 units and 1000 units, respectively. We include batch normalization and ReLU nonlinearity after conv1 and fc1. A dropout layer is added after fc1. We emphasize that the architecture of the encoder is purposefully chosen to be light weight.

%% adaptor training procedure

The dataset is split into training and testing based on the context, which is defined as the set of the target and distractor images (each trial consists of 3 distractor images and 1 target). The contexts of training examples will not overlap with the contexts of test examples. In practice, we have a 75\%, 25\% split ratio.

The binary classification task is to predict whether or not a given sketch is of a given image. Positive examples come from pariticipants' sketches and their corresponding target images. Negative samples come from pairing a participant's sketch with a randomly chosen image from the set of all images excluding the correct image. We note that training and testing sets will include examples from both close- and far-contexts. This training procedure closely resembles training Word2Vec with skip-gram and negative sampling where there is a similar motivation of learning shared embedding spaces.

During training, the model is shown minibatches of tuples, where each tuple contains $(r_{1}, s_{1})$, $(r_{2}, s_{2})$, $(r_{1}, s_{2})$, $(r_{2}, s_{1})$ where $r_{1}$, $r_{2}$ are two rendered images, and $(s_{1}, s_{2})$ are two sketches i.e. a minibatch of 10 would contain 40 examples. Each tuple is as follows: choose the next available image $r_{1}$, find its corresponding sketch $s_{1}$; choose 1 of the 3 distractor images from the same trial as $r_{2}$; its corresponding sketch is $s_{2}$. In each tuple, the first two examples are positive and last two examples are negative. The intuition behind this design is to balance the classes shown to the model at each gradient step.

Given an image $r_{i}$, a sketch $s_{i}$ and the label $l_{i}$, we define the objective function $J$ as

\begin{multline}
    J(r_{i}, s_{i}) = -l_{i} \textup{ log}(\sigma(\textup{corr}(f(r_{i}), f(s_{i})))) \\ - (1 - l_{i}) \textup{ log}(1 - \sigma(\textup{corr}(f(r_{i}), f(s_{i}))))
\end{multline}

where $f$ is the visual encoder network, $\sigma$ is a sigmoid transformation, and $\textup{corr}$ is pearson correlation. We train the model using stochastic gradient descent via Adam with a learning rate of $0.001$, a batch size of $25$ for $20$ epochs. We use early stopping based on a small validation set. The trained model has 73.6\% classification accuracy on the held-out test set, suggesting some generalization to unseen contexts.


\subsubsection*{Social reasoning and decision-making module}

We compare an \emph{asocial sketcher} $S_0$ that simply aims to make a drawing with high perceptual similarity to the target against a \emph{social sketcher} $S_1$ that takes into account context. To instantiate communicative flexibility in context, we generalize the Rational Speech Act (RSA) modeling framework, originally developed for linguistic communication. At the core of the RSA framework, as applied to linguistic communication, is the Gricean maxim of quantity: to make your contribution as informative as is required by context but no more so. In particular,

Formally, the sketcher, $\mathcal{S}$, is a decision-theoretic agent who produces drawings, $d$, by (soft) maximizing their utility function, $U$, given a particular object referent, $o$:

\begin{equation}
\mathcal{S}(d|o) \propto e^{\alpha U(d,o)}
\end{equation}

Here, $\alpha$ is a softmax parameter. As $\alpha \rightarrow \infty$, the sketcher approaches the optimal maximizing policy.

The sketcher's utility function, $U$, trades off the extent to which the drawing is informative to the viewer, $\mathcal{V}$, and the cost of producing the drawing, $C(d)$. This notion of informativity is defined by the (natural log) probability that a viewer would select the true object referent given the drawing and current context:

\begin{equation} \label{sketcher_utility}
U(d, o) = \ln \mathcal{V}(o|d) - C(d)
\end{equation}

This sketcher $\mathcal{S}$ is \emph{pragmatic} because they evaluate informativity by considering rational viewer agent ($\mathcal{V}$) who assigns a score to the correspondence between the drawing and each object that is proportional to the perceptual similarity, $P$, between the drawing and each potential object referent in context:

\begin{equation} \label{literal_viewer_score}
\mathcal{V}(o|d) \propto P_{o,d}
\end{equation}

To compute these perceptual similarity scores, the sketcher agent (who can also view the sketch and objects) extracts a feature vector representation from each of the target and distractor objects using the fixed visual encoder, and computes the distances between each of these vectors and the potential sketches it could produce.

This has the effect of biasing the model to depict properties of the target object that distinguish it from the distractors, while also preferring sketches that are less costly to produce, $C(d)$, i.e., requiring fewer strokes or less ink.

}

\showmatmethods % Display the Materials and Methods section

\acknow{Thanks to XX, XX, XX for helpful comments and discussion.}

\showacknow{} % Display the acknowledgments section

% \pnasbreak splits and balances the columns before the references.
% Uncomment \pnasbreak to view the references in the PNAS-style
% If you see unexpected formatting errors, try commenting out \pnasbreak
% as it can run into problems with floats and footnotes on the final page.
%\pnasbreak

% Bibliography
\bibliography{references}

\end{document}
