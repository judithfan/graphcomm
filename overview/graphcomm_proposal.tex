\documentclass[12pt]{article}
\usepackage{ebgaramond}
\usepackage{apacite}
\usepackage[protrusion=true,expansion=true]{microtype} 
\usepackage{fullpage}

\title{\textbf{Modeling Graphical Communication}}

\author{Judith Fan, \textit{with Robert Hawkins \& Justine Kao}}

\begin{document}
\maketitle % Print the title section

\section{Motivation}

\subsection{Why graphical communication?}

Drawing is a powerful tool for communication --- with just a few strokes it is possible to convey the identity of a face \cite{bergmann2013impact} or express an intention to act \cite{Galantucci:2005uh}. Moreover, drawn images predate the historical record, are pervasive in human culture, and are often produced prolifically in childhood.

Examining graphical communication provides a window into how meaningful tokens emerge from individual experience and social interactions, especially when conventionalized carriers of meaning are not available (e.g., words). Over the long-term, integrating insights across studies of linguistic and graphical modes of communication may yield general principles that govern how people negotiate shared meanings.  

\subsection{How drawing influences object representations}

In ongoing work, Fan, Yamins, and Turk-Browne (in prep) are investigating the basis for the ability to communicate an idea by drawing, by evaluating the premise that our ability to recognize objects and produce recognizable drawings of objects are linked by a common internal substrate --- a generalized object representation. We found that a convolutional neural network whose architecutre mirrors that of the ventral visual stream we was optimized to handle natural image variation in photographs \cite{Yamins:2014gia} was able to generalize to simple sketches of objects, without having to posit new specialized mechanisms. This suggests that the capacity for visual abstraction may be rooted in the functional architecture of the visual system.

Moreover, this result allows us to use this model as a tool to probe representational change: how does learning to draw objects influences the representation of object categories in the mind? Insofar as practice drawing allows one to learn the diagnostic features that better communicate the identity of an object (e.g., horse), they hypothesized that such practice might improve people's ability to make recognizable drawings of other, visually similar objects (e.g., sheep, cow). To test this, they designed a training paradigm in which participants repeatedly sketched some objects, while the model guessed the identity of the sketched object, providing trial-by-trial feedback. They found that repeatedly sketched objects were better recognized after training, while --- surprisingly --- sketches of unpracticed but similar objects worsened. These results show that visual production can reshape the representational space for objects in various ways: by differentiating trained objects and merging other nearby objects in the space.

\subsection{How do communicative interactions establish shared knowledge?}

How does social context constrain how ideas are expressed in drawings? That is, in addition to capturing physical properties of objects, successful drawings may depend on accurately representing the knowledge state of the viewer and knowing what knowledge is shared. Such shared knowledge need not be assumed, but may be established in the course of interactions between communication partners.

From \citeA{Brennan:1996ud} and others, we know that a individual's choice of label for an object is sensitive to the shared history of interactions with their interlocutor. Sometimes this results in the use of labels that might not be maximally efficient/distinctive in a given circumstance, but match the label previously used to refer to that same object (a ‘conceptual pact’ about the object-label pairing). 

Additionally, prior work on graphical dialogue suggests that pairs of individuals who have interacted before tend to produce drawings that are more `abstract' and similar to one another than individuals who have had just as much experience with the drawing task, but previously interacted with other individuals \cite{Healey:2007vq}. Moreover, pairs that are able to `mutually-modify' each other's drawings in real time (as opposed to taking turns) also end up producing drawings with similar styles and are more abstract. However, a deeper understanding of the mechanisms by which this form of interaction shapes the representational content of communicative drawings has been limited by the methodologies employed in these earlier studies, which relied on relatively coarse measures and qualitative analysis of response variables. 

\section{Proposal}

\subsection{How does interaction history guide graphical communication?}

The goal of the proposed experiment is twofold: to reveal how people negotiate graphical conventions for expressing certain concepts through repeated interactions, and to understand the general consequences of such interaction history on subsequent communication of novel concepts. 

To accomplish this, we plan to develop a graphical-dialogue paradigm in which one participant (Artist) has the goal of producing drawings of objects that the other participant (Viewer) will be able to identify. To investigate the importance of interaction history, we will manipulate how much and what kinds of objects they practice drawing/identifying before they are tested on a set of novel objects.

\textit{[What follows is a concrete experiment proposal inspired by these general aims, the specific details of which are of course subject to various modifications.]}

\subsection{Stimuli}

Objects will be sampled from a set of four real-world object categories defined at the basic level: dogs, birds, cars, and airplanes. Each category will contain eight exemplars. 

Images of these objects will serve as cues in the drawing task, as well as alternatives in the identification task. These images will be rendered from 3D mesh models, permitting fine control over low-level visual properties of these images, including size, viewpoint, and illumination. 

\subsubsection{Design and Procedure}

Adopting a similar approach to Fan, Yamins, and Turk-Browne (in prep), we plan to employ an experimental design that probes generalization by contrasting test performance on untrained objects that belong to the same category as trained objects with performance on untrained objects that belong to a different category. 

For each session, we will recruit two naive participants. 

During an initial learning phase, each participant will take turns as Artist and Viewer, alternating roles on each trial. They will practice on the same subset of objects (4 of 8) sampled from half of the categories (2 of 4). Each participant will have an opportunity to draw each object twice, resulting in 32 total learning trials, comprising 16 drawing trials and 16 guessing trials for each participant. These trials may be blocked by category, such that all the trials in the same block involve objects from the same category (4 blocks, 2 per category). 

On each trial of this practice phase, the Artist will be cued with an image of one of these objects (target) and prompted to produce a quick sketch of it (<=30s). This sketch will then be revealed to the Viewer, whereupon he/she will guess which object the sketch corresponds to by selecting among all 8 alternatives from the target category. These alternatives will be displayed as images depicting each of the trained objects from identical viewpoints. After the Viewer makes a guess, the pair will receive feedback: indicating whether the choice was correct or not; if incorrect, the Viewer will be shown which object was the target. 

Immediately following the practice phase, the test phase will begin. During this phase, participants will attempt to communicate the remaining objects in the trained categories (8 total, 4 from each category), as well as a subset of objects (4 of 8) sampled from the held-out categories (2 of 4). 

Each participant will be asked to 



32 objects total

4 objects from 2 categories are trained (8 trained objects overall)

Test on held-out objects (remaining 8 objects from trained categories + 8 randomly sampled objects in other categories)

So 32 training trials (2x per object per participant) = 2 x 8 x 2

16 test trials without feedback. 

\subsubsection{Predictions}

\section{Future directions}

\subsection{Emergence of novel graphical conventions}

%  Expressivity and compressibility \cite{Kirby:2015gi}

In some cases, verbal and visual conventions for referring to objects may not be well established (e.g., music drawing task from \citeNP{Healey:2007vq}), or not available at all (cf. coordination game in \citeNP{Galantucci:2005uh}). For instance, auditory textures \cite{McDermott:2013ky} lack the rhythmic and pitch information that are captured in standard musical notation, and may be hard to describe. Thus, someone who wants to visually convey which auditory texture they are listening to must innovate a graphical scheme for capturing its identifying physical features. While the initial selection of encoding schemes may be highly underdetermined, repeated interactions between communication partners may induce convergence upon a consistent way of referring to their shared experiences. Understanding the dynamics by which such interactions yield convergence is a key target for this work. 

\subsection{Coping with knowledge asymmetries in graphical communication}

Sometimes, two people interacting with one another may have access to non-overlapping knowledge \cite{Wu:2007tz}. For example, in teacher-student interactions, the teacher is presumed to have greater knowledge about the topic at hand than the student. In such cases, how does the teacher flexibly adjust their output in order to accommodate this discrepancy in knowledge (or fail to do so)? Understanding what allows a teacher (or any person more informed about a subject than their interlocutor) to succesfully overcome these challenges, by taking into account what the student already knows/believes, is a key challenge for this work. To experimentally manipulate discrepancies in knowledge, one member of a pair may be given additional examples during training, extended opportunities to practice, or richer feedback, prior to interacting with a less-experienced partner. 

\bibliographystyle{apacite}
\setlength{\bibleftmargin}{.125in}
\setlength{\bibindent}{-\bibleftmargin}
\bibliography{references.bib}

\end{document}
