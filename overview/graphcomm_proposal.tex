\documentclass[12pt]{article}
\usepackage{ebgaramond}
\usepackage{apacite}
\usepackage[protrusion=true,expansion=true]{microtype} 
\usepackage{fullpage}

\title{\textbf{Modeling Graphical Communication}}

\author{Judith Fan, Robert Hawkins, Justine Kao}

\begin{document}
\maketitle % Print the title section

\section{Why graphical communication?}

Drawing is a powerful tool for communication --- with just a few strokes it is possible to convey the identity of a face \cite{bergmann2013impact} or express an intention to act \cite{Galantucci:2005uh}. Moreover, drawn images predate the historical record, are pervasive in human culture, and are often produced prolifically in childhood.

Examining graphical communication provides a window into how meaningful tokens emerge from individual experience and social interactions, especially when conventionalized carriers of meaning are not available (e.g., words). Over the long-term, integrating insights across studies of linguistic and graphical modes of communication may yield general principles that govern how people negotiate shared meanings.  

\section{How drawing influences object representations}

In ongoing work, Fan, Yamins, and Turk-Browne (in prep) are investigating the basis for the ability to communicate an idea by drawing, by evaluating the premise that our ability to recognize objects and produce recognizable drawings of objects are linked by a common internal substrate --- a generalized object representation. They found that a convolutional neural network whose architecutre mirrors that of the ventral visual stream and was optimized to handle natural image variation in photographs \cite{Yamins:2014gia} was able to generalize to simple sketches of objects, without having to posit new specialized mechanisms. This suggests that the capacity for visual abstraction may be rooted in the functional architecture of the visual system.

Moreover, this result set them up to use this model to ask: how does learning to draw objects influences the representation of object categories in the mind? Perhaps practice drawing allows one to learn the diagnostic features that better communicate the identity of an object (e.g., horse), and this practice might improve people's ability to make recognizable drawings of other, visually similar objects (e.g., sheep, cow). To test this, they designed a training paradigm in which participants repeatedly sketched some objects, while the model guessed the identity of the sketched object, providing trial-by-trial feedback. They found that repeatedly sketched objects were better recognized after training, while sketches of unpracticed but similar objects worsened. These results show that visual production can reshape the representational space for objects in various ways: by differentiating trained objects and merging other nearby objects in the space.

\section{How do communicative interactions establish shared knowledge?}

However, drawing (like speaking or writing) can communicate different meanings in different contexts. How does social context constrain how ideas are expressed in drawings? From \citeNP{Brennan:1996ud} and others, we know that a individual's choice of label for an object is sensitive to the shared history of interactions with their interlocutor. Sometimes this results in the use of labels that might not be maximally efficient/distinctive in a given circumstance, but match the label previously used to refer to that same object (a ‘conceptual pact’ about the object-label pairing). 

Additionally, prior work on graphical dialogue suggests that pairs of individuals who have interacted before tend to produce drawings that are more `abstract' and similar to one another than individuals who have had just as much experience with the drawing task, but previously interacted with other individuals \cite{Healey:2007vq}. Moreover, pairs that are able to `mutually-modify' each other's drawings in real-time (as opposed to taking turns) also end up producing drawings with similar styles and are more abstract. However, a deeper understanding of the mechanisms by which this form of interaction shapes the representational content of communicative drawings has been limited by the methodologies employed in these earlier studies, which relied on relatively coarse measures and qualitative analysis of response variables. 

\section{Proposal: How does interaction history guide graphical communication?}

\subsection{Stimuli}

\subsection{Design and Procedure}

\subsection{Predictions}

\section{Future directions}

\subsection{Emergence of novel graphical conventions}

In some cases, verbal and visual conventions for referring to objects may not be well established (e.g., music drawing task from \citeNP{Healey:2007vq}), or not available at all (cf. coordination game in \citeNP{Galantucci:2005uh}). For instance, auditory textures \cite{McDermott:2013ky} lack the rhythmic and pitch information that are captured in standard musical notation, and may be hard to describe. Thus, someone who wants to visually convey which auditory texture they are listening to must innovate a graphical scheme for capturing its identifying physical features. While the initial selection of encoding schemes may be highly underdetermined, repeated interactions between communication partners may induce convergence upon a consistent way of referring to their shared experiences. Understanding the dynamics by which such interactions yield convergence is a key target for this work. 

% Tests of generalization would be a powerful way to uncover the structure and scope of these initial proposals for encoding schemes. When later confronted with these targets in a novel context (e.g., broader or narrower than during initial exposure), how does this prior experience influence how these targets are drawn? How does prior experience with some targets influence how novel, yet similar targets are drawn?

% One possibility is that people will be so heavily biased by their prior interactions that they will only be able to recycle previously established tokens to refer to new targets (strong form of `conceptual pact'). A second possibility is that people do not rely on past experience whatsoever, and instead devise entirely new ways of referring to new targets (no effect of history). 

% A third scenario is one in which people exploit conventions established during initial interactions and extend them to accommodate new targets, such as by merging elements of previously used tokens, or by introducing new elements only to handle salient new features of test targets. 

% Psychological similarity between test and training targets provides a reasonable constraint on this generalization: a test target that is similar to one of the initially trained targets might be represented more similarly to the original; a more complex test target that is similar to multiple initially trained targets might be represented by recombining features from multiple members of the training set. 

% Moreover, pairs of participants that were trained on a relatively narrow set of examples may exhibit different patterns of behavior at test than participants trained on a broad set of examples. For instance, narrow training may promote overfitting and reduced ability to generalize, whereas broad training may lead to greater exploratory variation during training (potentially slowing down initial entrainment), but promotes generalization. 

\subsection{Coping with knowledge asymmetries in graphical communication}

Sometimes, two people interacting with one another may have access to non-overlapping knowledge \cite{Wu:2007tz}. For example, in teacher-student interactions, the teacher is presumed to have greater knowledge about the topic at hand than the student. In such cases, how does the teacher flexibly adjust their output in order to accommodate this discrepancy in knowledge (or fail to do so)? Understanding what allows a teacher (or any person more informed about a subject than their interlocutor) to succesfully overcome these challenges, by taking into account what the student already knows/believes, is a key challenge for this work. To experimentally manipulate discrepancies in knowledge, one member of a pair may be given additional examples during training, extended opportunities to practice, or richer feedback, prior to interacting with a less-experienced partner. 

\bibliographystyle{apacite}
\setlength{\bibleftmargin}{.125in}
\setlength{\bibindent}{-\bibleftmargin}
\bibliography{references.bib}

\end{document}
