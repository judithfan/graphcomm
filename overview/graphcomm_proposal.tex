\documentclass[12pt]{article}
\usepackage{ebgaramond}
\usepackage[protrusion=true,expansion=true]{microtype} 
\usepackage{fullpage}

\title{\textbf{Modeling Graphical Communication}}

\author{Judith Fan, Robert Hawkins, Justine Kao}

\begin{document}
\maketitle % Print the title section

\section{Why graphical communication?}

Drawing is a powerful tool for communication --- with just a few strokes it is possible to convey the identity of a face \cite{bergmann2013impact}or express an intention to act \cite{Galantucci:2005uh}. Moreover, drawn images predate the historical record, are pervasive in human culture, and are often produced prolifically in childhood. 

Examining graphical communication provides a window into how meaningful tokens emerge from experience, especially when conventionalized carriers of meaning are not available (e.g., words). Over the long-term, integrating insights across studies of linguistic and graphical modes of communication may yield general principles that govern how people negotiate shared meanings.  

\section{What sort of shared knowledge is established in the course of communicative interactions?}

\subsection{Background}

From Brennan and Clark (1996) and others, we know that a person’s choice of label for an object is sensitive to the shared history of interactions with their interlocutor. Sometimes this results in the use of labels that might not be maximally efficient/distinctive in a given circumstance, but rather the same label previously used to refer to that same object (a ‘conceptual pact’ about the object-label pairing). 

In ongoing work (Fan, Yamins, and Turk-Browne, in prep), we’re examining how label feedback in an object-drawing task influences how people subsequently draw those objects to more effectively convey its identity. How does such practice with communicative drawing affect features of those subsequent drawings? Presently, we interpret better production performance as reflecting the ability of people to home in visual features most diagnostic of object identity over time. 

However, drawing (like speaking or writing) can communicate different meanings in different contexts. How does social/conceptual context constrain visual production? For example, in the context of a drawing task where the targets belong to different basic-level categories (e.g., dogs, furniture, flowers), what features do people pick out to include in their drawings vs. when the targets are exemplars of the same basic-level category (e.g., pomeranian vs. dachshund)? 


\subsection{Proposal}

In some cases, verbal and visual conventions for referring to objects may not be well established (e.g., music drawing task from Healey, et al., 2001), or not available at all (cf. coordination game in Galantucci, 2005). For instance, auditory textures (McDermott and Simoncelli, 2013) lack the rhythmic and pitch information that are captured in standard musical notation, and may be hard to describe. Thus, someone who wants to visually convey which auditory texture they are listening must innovate a graphical scheme for capturing its identifying physical features. Initially, these encoding schemes are likely to be quite variable; over time, however, variability may decrease as the participants in the interaction entrain to each other, and converge upon a consistent set of identifiers. Understanding the dynamics by which these interactions induce such convergence is a key target for this work. 

Tests of generalization would be a powerful way to uncover the structure and scope of these initial proposals for encoding schemes. When later confronted with these targets in a novel context (e.g., broader or narrower than during initial exposure), how does this prior experience influence how these targets are drawn? How does prior experience with some targets influence how novel, yet similar targets are drawn?

One possibility is that people will be so heavily biased by their prior interactions that they will only be able to recycle previously established tokens to refer to new targets (strong form of `conceptual pact'). A second possibility is that people do not rely on past experience whatsoever, and instead devise entirely new ways of referring to new targets (no effect of history). 

A third scenario is one in which people exploit conventions established during initial interactions and extend them to accommodate new targets, such as by merging elements of previously used tokens, or by introducing new elements only to handle salient new features of test targets. 

Psychological similarity between test and training targets provides a reasonable constraint on this generalization: a test target that is similar to one of the initially trained targets might be represented more similarly to the original; a more complex test target that is similar to multiple initially trained targets might be represented by recombining features from multiple members of the training set. 

Moreover, pairs of participants that were trained on a relatively narrow set of examples may exhibit different patterns of behavior at test than participants trained on a broad set of examples. For instance, narrow training may promote overfitting and reduced ability to generalize, whereas broad training may lead to greater exploratory variation during training (potentially slowing down initial entrainment), but promotes generalization. 

\section{How does knowing what is not shared knowledge influence communication? }

This is where it gets really juicy. Sometimes, two people interacting with one another may have access to non-overlapping knowledge (cf. Wu and Keysar, 2007). For example, in teacher-student interactions, the teacher is presumed to have greater knowledge about the topic at hand than the student. In such cases, how does the teacher flexibly adjust their output in order to accommodate this discrepancy in knowledge (or fail to do so)? Understanding what allows a teacher (or any person more informed about a subject than their interlocutor) to succesfully overcome these challenges, by taking into account what the student already knows/believes, is a key challenge for this work. 

To experimentally manipulate discrepancies in knowledge, one member of a pair may be given additional examples during training, extended opportunities to practice, or richer feedback, prior to interacting with a less-experienced partner. 

\bibliographystyle{apacite}
\setlength{\bibleftmargin}{.125in}
\setlength{\bibindent}{-\bibleftmargin}
\bibliography{references.bib}

\end{document}
